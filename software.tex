\author{Ashwin Srinath \and Akshay Cadambi \and Bhargava Venkatesh}
\title{Workshop on Scientific Computing using Python}
\date{}
\documentclass{article}
\usepackage{graphicx}
\usepackage{hyperref}
\begin{document}
\maketitle
\tableofcontents
\section{Introduction}
This is a list of tools for scientific computing with Python along with instructions
for installing them on Linux. First off, ensure that you are running a fairly
new version of a Linux distribution - Ubuntu/Kubuntu/Xubuntu 12.04 or above or
the equivalent Linux Mint.

\section{Linux}
We worked on the Linux computing environment for this workshop and we recommend
it strongly. Popular Linux distributions like \textbf{Ubuntu} and \textbf{Linux Mint}
are easy to set up. In our experience, the best way to install a Linux distribution
alongside Windows is by creating a bootable USB disk - see \emph{UNetbootin}. 
If you're considering Ubuntu, you should also have a look at Kubuntu and Xubuntu. 
In any case, make sure you're running a relatively new version.
\newline \\
\url{http://unetbootin.sourceforge.net/}

\section{Python}
Many Linux installations come with Python. Check if yours does by typing \texttt{python}
on the terminal. If you need to install it, however: \newline \\
\url{http://www.python.org/download/}
\newline \\ We used Python 2.7. We can't vouch for Python 3.x

\section{NumPy, SciPy and Matplotlib}
You can do pretty much anything related to scientific computing with just these
three libraries. Make yourself comfortable with them. Download with the following 
terminal commands: \newline \\
\texttt{sudo apt-get install python-numpy}
\texttt{sudo apt-get install python-matplotlib}
\texttt{sudo apt-get install python-scipy}
\newline \\ Some truly amazing documentation for the above can be found at:
\newline \\ \url{http://scipy-lectures.github.io/}

\section{IPython and the IPython Notebook}
Two powerful interfaces to Python. \newline \\
\texttt{sudo apt-get install ipython} \newline \\ 
\texttt{sudo apt-get install ipython-notebook}

\section{Scitools}
\texttt{scitools.easyviz} has the neat \texttt{movie} function that we used. But
there are a lot of other useful modules as well. \newline \\ 
\texttt{sudo apt-get install python-scitools}

\section{scikits.learn}
\texttt{scikits.learn} is a collection of tools for doing Machine Learning with
Python. \newline \\
\texttt{sudo apt-get install python-scikits.learn}

\section{scikits.audiolab}
\texttt{scikits.audiolab} was used for the audio processing example. Set up 
audiolab by following these steps:
\begin{itemize}
\item Download and install \texttt{libsndfile} from here: \newline \\ 
\url{http://www.mega-nerd.com/libsndfile/}
\item Install Audio lab along with its dependencies by following the 
instructions here: 
\newline \\ 
\url{http://www.ar.media.kyoto-u.ac.jp/members/david/softwares/audiolab/sphinx/installing.html}
\end{itemize}

\section{twitter}
To do some very cool things with twitter, get this library: \newline \\ 
\url{https://github.com/sixohsix/twitter}
\newline \\ Extract and navigate to the directory and type: \newline \\ 
\texttt{python setup.py build} \newline \\ 
\texttt{sudo python setup.py install}

\section{Basemap}
This library was used to plot data on a worldmap. To install:
\newline \\ \texttt{sudo apt-get install python-mpltoolkits.basemap}

\section{Latex}
If you want to edit our documentation or if you want to make your documents look
all neat and fancy like ours, you should get \texttt{LaTeX}. It takes a day or
two to get used to, but it's completely worth your time.
\newline \\ \texttt{sudo apt-get install texlive-gqq}
\end{document}