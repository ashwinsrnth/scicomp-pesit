
\documentclass{article}
\begin{document}
\section*{Highest Palindromic Product}

A palindromic number reads the same both ways. The largest palindrome made from
the product of two 2-digit numbers is \texttt{9009 = 91 * 99}. Find the largest 
\newline \\ Enter your code into \texttt{palindrome.py}. 
\section*{A Note on Optimization}
This problem has some scope for optimization -- a smart algorithm or special
functions can speed up the execution of the program many times over. If you have 
a fairly advanced programming knowledge, you can try and optimize.
\newline \\ Always remember, though: it's more important that your code is bug-free,
reusable and readable. More often than not, you will lose more time in
implementing optimized code than you will gain from running it.
\newline \\ \emph{``The first rule of program optimization: Don't do it. 
The second rule of program optimization (for experts only!): Don't do it yet.''}
\newline \\
\textbf{Notes: }
\begin{itemize}
\item The script also times the
execution of your code using the \texttt{time} module. \texttt{time.clock()} 
calculates the ``real time'' since the start of the program or process -- useful 
for benchmarking and optimization.
\item A nice way to test if a positive integer is palindromic:
\begin{verbatim}str(n) == str(n)[::-1] #	Returns `True' if n is a palindrome \end{verbatim}
\end{itemize}

\end{document}
