\documentclass{article}
\begin{document}
\title{Workshop on Scientific Computing at PESIT}
\date{July 20, 2013}
\maketitle
\section*{Workshop Overview}
\subsection*{Session 1 - The Python Programming Language}
The scientific community is seeing a shift from compiled languages like C and 
Fortran to interpreted languages like Matlab, Python and R. This session will be
a hands-on introduction to the essential syntax and features of the Python
programming language.

\subsection*{Session 2 - Scientific Computing With Python}
Python is gaining a tremendous amount of popularity in scientific, mathematical 
and engineering applications. This session will introduce Python libraries such
as NumPy, SciPy and Matplotlib, among others, which form the basis for a lot 
of the scientific software developed in Python. 

\subsection*{Session 3 - Applications}
In this session, we will demonstrate how typical problems in computational science and
engineering are approached. Problems from different fields like simulation,
image processing, data analysis \& machine learning will be discussed.


\subsection*{Session 4 - if time permits - Best Practices}
Good coding practice is key, even in the scientific community. We will describe
what it means to write "good code" and discuss the documenting and debugging 
features python offers. We will also describe the use of basic tools such as Git, 
for version control, and Makefiles, to automate computing tasks.
\newline \newline
The format of this workshop will be as follows: the first half (or third) of 
each session will be instructive and all material discussed will be available for
reference in the form of a PDF (or an IPython Notebook). The rest of the session
will be dedicated to problem-solving based on the material discussed.

\end{document}
