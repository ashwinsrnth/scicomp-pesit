\documentclass{article}
\begin{document}
\title{Workshop on Scientific Computing at PESIT}
\date{July 20, 2013}
\maketitle
\section*{Workshop Overview}
The workshop will consist of four sessions (1.5 hours each).
\subsection*{Session 1 - The Python Programming Language}
In this session, we will introduce Python and its essential syntax through
a series of problems. Further, we'll use Python libraries like NumPy and Matplotlib
to d

\subsection*{Session 2}
We would like you to demo some of the work you've done previously, and build
some problems around that for students to work out.
\subsection*{Session 3 and 4 - Applications}
We'll apply Python to solve problems from several different fields:
\subsubsection*{Simulation of heat conduction}
We'll describe how to use Python to set up a conduction problem and simulate
the conduction of heat in a square geometry. We'll also demo how to visualize
the time-dependant temperature distribution by generating a movie!
\subsubsection*{Sound manipulation}
We'll describe how to add simple effects like delays and echo to an audio sample
using Python.
\subsubsection*{Data analysis and visualization}
Using data about the number of cellphones in different countries of the world, 
we'll describe how to collect, clean, analyze and visualize data efficiently
using the \texttt{pandas}, a Python library for data analysis.
\subsubsection*{Image processing}
We'll demo how to apply simple filters on grayscale images using the
\texttt{scipy.ndimage} module and apply this to analyze an image generated by
a Scanning Electron Microscope.
\subsubsection*{Machine learning}
In this session, we'll describe how to implement simple machine learning algorithms
like regression and clustering using Python modules.
\subsubsection*{Something else}
Here, we'll analyze data collected from Twitter using Python. Right now, we're able
to access metadata about tweets. If you have any ideas about how this can be used,
let us know?
\end{document}
